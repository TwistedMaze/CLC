


\documentclass{beamer}

\usepackage{babel}
\usepackage{hyperref}
\usetheme{Dresden}

\title{Mental Comparisons}
\subtitle{Ciências da Linguagem e da Cognição}
\author{Filipe Onofre, 51737 \linebreak Madalena Tomás, XXXXX}

\begin{document}
\begin{frame}
	\titlepage
\end{frame}
\begin{frame}
	\frametitle{Índice}
	\tableofcontents
\end{frame}

\section{Introdução Teórica}

\begin{frame}
	\frametitle{Comparações Mentais}
	\begin{description}
		\item Comparações Mentais, ou Julgamentos Comparativos, são uma forma comum de processo cognitivo na rotina diária do ser humano.
		\item Comparações mentais são julgamentos de dimensão de interesse.
		\item Tipicamente estudado comparando tamanhos de animais.
		\item O estudo prévio de comparações originou três descobertas tidas históricamente como padrão. 
	\end{description}
\end{frame}
\begin{frame}
	\frametitle{Comparações Mentais}
	\framesubtitle{Efeito de Distância Simbólica}
	Questionando individuos sobre a diferença de tamanho entre dois animais, Moyer conclui que:
	\begin{itemize}
	\item Quanto maior a diferença de tamanho entre os animais, mais rápida é a resposta dos inquiridos.
	\item 	Moyer e Bayer designaram o resultado de Efeito da Distância Simbólica
	\item Moyer \& Bayer, (1976).
	\end{itemize}
\end{frame}
\begin{frame}
	\frametitle{Comparações Mentais}
	\framesubtitle{Efeito de congruência}
	
	Quando questionamos sobre o tamanho de quaisquer dois objetos, Wallis e Audley observaram que:
	
	\begin{itemize}
	\item é mais rápida a resposta de "qual o mais pequeno" se ambos objetos forem pequenos.
	
	\item se ambos os objetos forem grandes, demora mais tempo distinguir o mais pequeno
	
	\item Wallis \& Audley (1964) descreveram o fenómeno como efeito de Congruência
	\end{itemize}
\end{frame}
\begin{frame}
	\frametitle{Comparações Mentais}
	\framesubtitle{Efeito dos extremos}

	Leth-Steenson e Marley observaram empiricamente que:
	\begin{itemize}
		\item é mais rápida a distinção entre o tamanho de dois objetos ou animais quando as suas dimensões são polarmente opostas.
		\item é mais demorada a distinção entre dois objetos ou animais quando as suas dimensões são aproximadas
		\item Leth-Steenson \& Marley, (2000).
	\end{itemize}
\end{frame}
\begin{frame}
	\frametitle{Categorização e o Processo de Comparação}
	Čech e Shoben, em 2001 reveêm as provas de que a categorização interfere com o processo de comparação e com os efeitos anteriormente observados
	
	\begin{itemize}
		\item O Efeito de Distância Simbólica pode ser atenuado ou mesmo eliminado se os objetos comparados forem de diferentes categorias.(Sailor \& Shoben, 1993)
		\item Os inquiridos distinguem mais rápidamente o mais pequeno de um par de animais, quando animais de maior dimensão se encontram tambem presentes. (Čech \& Shoben,1985, Efeito de Congruência)
		\item Čech e Shoben argumentam que o Efeito dos Extremos se relaciona com a facilidade em categorizar algo quanto ao tamanho  		
	\end{itemize} 
\end{frame}
%\begin{frame}
%	\frametitle{Outras abordagens}
%\end{frame}
\section{Experiência}
\begin{frame}
	\frametitle{Exercicio}
\end{frame}
\begin{frame}
	\frametitle{Resultados}
\end{frame}
\end{document}