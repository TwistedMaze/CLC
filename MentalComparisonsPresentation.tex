


\documentclass{beamer}

\usepackage{babel}
\usepackage{hyperref}
\usetheme{Madrid}

\title{Mental Comparisons}
\subtitle{Ciências da Linguagem e da Cognição}
\author{Filipe Onofre, 51737 \linebreak Madalena Tomás, XXXXX}

\begin{document}
\begin{frame}
	\titlepage
\end{frame}
\begin{frame}
	\frametitle{Outline}
	\tableofcontents
\end{frame}

\section{Introdução Teórica}

\begin{frame}
	\frametitle{Comparações Mentais}
	\begin{description}
		\item Comparações Mentais, ou Julgamentos Comparativos, são uma forma comum de processo cognitivo na rotina diária do ser humano.
		\item Comparações mentais são julgamentos de dimensão de interesse.
		\item Tipicamente estudado comparando tamanhos de animais.
		\item O estudo prévio de comparações originou três descobertas tidas históricamente como padrão. 
	\end{description}
\end{frame}
\begin{frame}
	\frametitle{Comparações Mentais}
	\framesubtitle{Efeito de Distância Simbólica}
	Questionando individuos sobre a diferença de tamanho entre dois animais, Moyer conclui que:
	\begin{description}
	\item Quanto maior a diferença de tamanho entre os animais, mais rápida é a resposta dos inquiridos.
	\item 	Moyer e Bayer designaram o resultado de Efeito da Distância Simbólica
	(Moyer \& Bayer, 1976).
	\end{description}
\end{frame}
\begin{frame}
	\frametitle{Comparações Mentais}
	\framesubtitle{Efeito de congruência}
\end{frame}
\begin{frame}
	\frametitle{Comparações Mentais}
	\framesubtitle{Efeito dos extremos}
\end{frame}
\begin{frame}
	\frametitle{Categorização e o Processo de Compração}
\end{frame}
\begin{frame}
	\frametitle{Outras abordagens}
\end{frame}
\section{Experiência}
\begin{frame}
	\frametitle{Exercicio}
\end{frame}
\begin{frame}
	\frametitle{Resultados}
\end{frame}
\end{document}